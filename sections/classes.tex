\section{Classes}
    \vspace{-2mm}
	\lstinputlisting{snippets/class.h}\vspace{-2mm}

\subsection{Styleguide}
    \mylstil{public}, \mylstil{protected} and \mylstil{private} members should be declared in that order.

\subsection{Best practices}
    \begin{enumerate}
        \item Default ctor; further ctors if needed
        \item Every user-defined ctor should initialize all member variables
        \item label user-defined dtor with \mylstil{virtual}
        \item "rule of zero" (if possible): no user-defined ctors, dtors, copy-ctors, move-ctors, copy-assignment-operators, move-assignment-operators
    \end{enumerate}

\subsection{Ctors}
    \begin{tabularx}{\columnwidth}{@{}l X@{}}
        \underline{Name:}          &Identical to class name\\
        \underline{Returntype:}    &None! Not void!\\
        \underline{Params:}        &Any\\
                            &-- none: default ctor\\
                            &-- const reference to own class: \textbf{copy-ctor}\\
        \underline{Task:}          &Prepares/initializes class
    \end{tabularx}

\subsection{Dtors}
    \begin{tabularx}{\columnwidth}{@{}l X@{}}
        \underline{Name:}          &Identical to class name\\
        \underline{Returntype:}    &None! Not void!\\
        \underline{Params:}        &None\\
        \underline{Task:}          &Deallocate memory/resources
    \end{tabularx}
    Only one destructor per class. (In some special cases overloading is possible)\newline
    Automatically called when class is no longer needed.
    
\subsection{Visibility}
    \begin{tabularx}{\columnwidth}{@{}l X@{}}
        \underline{Public:}       & Visible to everyone (within class, in derived, wherever class is used)\\
        \underline{Protected:}    & Only visible to class and derived classes\\
        \underline{Private:}      & Only visible within class
    \end{tabularx}
    
    \subsubsection{Friend attribute}
    A method within a class can be declared as \mylstil{friend}. Said method is then globally accessible and can access private members of the class.

    \subsubsection{Getter- and Setter-methods}
    Getter- and Setter-methods are used to control access to private members of a class. 
    They are public methods that return or set the value of a private member. 
    They can be used to check the validity of the value to be set or to hide the implementation of the class (e.g. if the class is part of a library).
    In certain cases the setter method can be declared as protected, if it is only to be used by the base- and derived classes.

    \subsection{UML (U\textcolor{gray}{nified} M\textcolor{gray}{odeling} L\textcolor{gray}{anguage})}
    \begin{center}
        \begin{tikzpicture}
    \umlclass[x=-2.7]{Super}{
        -privateMember: int\\
        \#protectedMember: char\\
        +publicMember: bool
    }{
        +publicMethod(): void\\
        \umlvirt{+publicVirtualMethod(parameter: int): void}\\
        \umlstatic{+staticMethod(): void}\\
        -privateMethod(): bool\\
        \#protectedMethod(value: int): char\\
        +Super()\\
    }
%     \node at (-1.75, 0.5) (pm) {};
%     \node at (-1.75, -0.4) (pmet) {};
%     \draw (pm) -- (-3.5,0.5) node[anchor=east] {Name} -- (pmet);
%     \node at (0.25, 0.5) (pmty) {};
%     \node at (1.8, -0.4) (pmetty) {};
%     \draw (pmty) -- (3.5,0.5) node[anchor=west] {(return)Type} -- (pmetty);
    \umlclass[x=2.5]{Derived}{
        +publicMember: bool
    }{
        +publicMethod(): void\\
        +Derived()\\
    }
    \umlinherit[geometry=--]{Derived}{Super}
\end{tikzpicture}
    \end{center}
    \vspace{-2mm}
    \mylstil{static} methods can be called without an instance of a class. They can be called like \mylstbox{Super.staticMethod();}. \mylstbox{+publicVirtualMethod()} refers to a \textit{pure virtual} method.
    
\subsection{Const}
    If \mylstil{const} is used after a method declaration, the method is not allowed to change any member variables of the class.

    \mylstbox{int someFunction() const;}

\subsection{Inheritance}
	If a class is to be inherited from, its destructor needs to be defined as virtual.
	A classic example of inheritance looks as follows:
    \vspace{-1mm}
	\lstinputlisting{snippets/inheritance.h}\vspace{-1mm}
	Classes can be derived from as \mylstil{public}, \mylstil{protected} or \mylstil{private}. Default of \mylstil{class} is \mylstil{private}, default of \mylstil{struct} is \mylstil{public}.
	This changes visibility of its inherited methods.
    \begin{center}
        \begin{tabular}{@{}l@{} c c@{}}
            \toprule
            \textbf{Inheritance visibility} & \textbf{visibility in base class} & \textbf{visibility in derived class}\\\toprule
            public (default w/ \mylstil{struct})   &\begin{tikzpicture}[baseline=-0.55cm, remember picture]
                                    \begin{scope}[every node/.style={fill=backcolour, minimum width=2cm, minimum height=0.45cm, align=center}]
                                        \node (box1pub) {public};
                                        \node[below=0.1mm of box1pub] (box1prot) {protected};
                                        \node[below=0.1mm of box1prot] (box1priv) {private};
                                    \end{scope}
                                    \addvmargin{0.5mm}
                                \end{tikzpicture}
                                &\begin{tikzpicture}[baseline=-0.55cm, remember picture]
                                    \begin{scope}[every node/.style={fill=backcolour, minimum width=2cm, minimum height=0.45cm, align=center}]
                                        \node (box2pub) {public};
                                        \node[below=0.1mm of box2pub] (box2prot) {protected};
                                        \node[below=0.1mm of box2prot] (box2priv) {invisible};
                                    \end{scope}
                                    \addvmargin{0.5mm}
                                \end{tikzpicture}\\\midrule
            protected           &\begin{tikzpicture}[baseline=-0.55cm, remember picture]
                                    \begin{scope}[every node/.style={fill=backcolour, minimum width=2cm, minimum height=0.45cm, align=center}]
                                        \node (box3pub) {public};
                                        \node[below=0.1mm of box3pub] (box3prot) {protected};
                                        \node[below=0.1mm of box3prot] (box3priv) {private};
                                    \end{scope}
                                    \addvmargin{0.5mm}
                                \end{tikzpicture}
                                &\begin{tikzpicture}[baseline=-0.55cm, remember picture]
                                    \begin{scope}[every node/.style={fill=backcolour, minimum width=2cm, minimum height=0.45cm, align=center}]
                                        \node (box4pub) {\color{gray!50}public};
                                        \node[below=0.1mm of box4pub] (box4prot) {protected};
                                        \node[below=0.1mm of box4prot] (box4priv) {invisible};
                                    \end{scope}
                                    \addvmargin{0.5mm}
                                \end{tikzpicture}\\\midrule
            private (default w/ \mylstil{class})          &\begin{tikzpicture}[baseline=-0.55cm, remember picture]
                                    \begin{scope}[every node/.style={fill=backcolour, minimum width=2cm, minimum height=0.45cm, align=center}]
                                        \node (box5pub) {public};
                                        \node[below=0.1mm of box5pub] (box5prot) {protected};
                                        \node[below=0.1mm of box5prot] (box5priv) {private};
                                    \end{scope}
                                    \addvmargin{0.5mm}
                                \end{tikzpicture}
                                &\begin{tikzpicture}[baseline=-0.55cm, remember picture]
                                    \begin{scope}[every node/.style={fill=backcolour, minimum width=2cm, minimum height=0.45cm, align=center}]
                                        \node (box6pub) {\color{gray!50}public};
                                        \node[below=0.1mm of box6pub] (box6prot) {\color{gray!50}protected};
                                        \node[below=0.1mm of box6prot] (box6priv) {\footnotesize private / invisibel};
                                    \end{scope}
                                    \addvmargin{0.5mm}
                                \end{tikzpicture}\\\bottomrule
        \end{tabular}
    \end{center}
    \begin{tikzpicture}[overlay, remember picture, >={Triangle[width=1.5mm, length=1.5mm]}]
        \draw[->] (box1pub.east) -- (box2pub.west);
        \draw[->] (box1prot.east) -- (box2prot.west);
        \draw[->] (box1priv.east) -- (box2priv.west);

        \draw[->] (box3pub.east) -- (box4prot.west);
        \draw[->] (box3prot.east) -- (box4prot.west);
        \draw[->] (box3priv.east) -- (box4priv.west);

        \draw[->] (box5pub.east) -- (box6priv.west) node[rotate=-27, midway, below=-1mm,xshift=-1mm] {\footnotesize private};
        \draw[->] (box5prot.east) -- (box6priv.west);
        \draw[->] (box5priv.east) -- (box6priv.west) node[midway, below=-0.5mm, xshift=-1mm] {\footnotesize invisible};
    \end{tikzpicture}\vspace{-2mm}
\subsubsection{ctor- and dtor-chaining}
    Order of ctor calls: base class(es) first, then derived class(es).\newline
    Order of dtor calls: derived class(es) first, then base class(es).\newline
    Example:
    \vspace{-2mm}
    \lstinputlisting{snippets/ctorchaining.cpp}\vspace{-2mm}
    Ctors are automatically chained, but can be explicitly called as well with an initializer list like in the example.

\subsubsection{vtable}
    \vspace{-2mm}
    \lstinputlisting{snippets/vtable.h}\vspace{-2mm}
    Previous code snippet of polymorphic code produces following memory map:
    \begin{center}
        % \begin{tikzpicture}
%     \umlclass[type=abstract]{Dog}{
%         -name_: string
%     }{
%         +Dog(name: string)\\
%         +\umlvirt{sit():void}\\
%         +\umlvirt{bark(): void}\\
%         +\umlvirt{getName(): void}
%     }
%     \umlclass[y=-2.5]{Husky}{

%     }{
%         +Husky(name: string)\\
%         +sit(): void\\
%         +bark(): void\\
%     }
%     \umlinherit[geometry=--]{Husky}{Dog}
%     \node[right=1.5cm of Dog, style={inner sep=0,outer sep=0}] (DogVtable) {%
%         \begin{tabular}{@{}c c@{}}
%             \toprule
%             {\bf\sffamily Method} &{\bf\sffamily Implementation}\\\toprule
%             {\sffamily sit()} &{\sffamily NULL}\\
%             {\sffamily bark()} &{\sffamily Dog::bark()}\\
%             {\sffamily getName()} &{\sffamily Dog::getName()}\\\bottomrule
%         \end{tabular}};
%     \draw[dashed, dash pattern=on 1mm off 1mm] (Dog) -- (DogVtable);
%     \node[right=1.5cm of Husky, style={inner sep=0,outer sep=0}] (HuskyVtable) {%
%         \begin{tabular}{@{}c c@{}}
%             \toprule
%             {\bf\sffamily Method} &{\bf\sffamily Implementation}\\\toprule
%             {\sffamily sit()} &{\sffamily Husky::sit()}\\
%             {\sffamily bark()} &{\sffamily Husky::bark()}\\
%             {\sffamily getName()} &{\sffamily Dog::getName()}\\\bottomrule
%         \end{tabular}};
%     \draw[dashed, dash pattern=on 1mm off 1mm] (Husky) -- (HuskyVtable);
% \end{tikzpicture}

\begin{tikzpicture}[font=\sffamily\small]
    \node[draw, minimum width=1.5cm, minimum height=0.3cm] (petvt0) {};
    \node[draw, minimum width=1.5cm, minimum height=0.3cm, above=-0.1mm of petvt0] (petvt1) {};
    \node[left=1mm of petvt0] {0};
    \node[left=1mm of petvt1] {1};
    \node[below=0mm of petvt0.south west] {\bfseries pet};
    
    \node[draw, minimum width=1.5cm, minimum height=0.3cm, right=1.5cm of petvt1] (hskymm1) {};
    \node[draw, minimum width=1.5cm, minimum height=0.3cm, inner sep=0.7mm, anchor=west, below=-0.1mm of hskymm1] (hskymm0) {\bfseries "don"};
    
    \node[draw, minimum width=1.5cm, minimum height=0.3cm, below=0.5cm of hskymm0] (dogmm1) {};
    \node[draw, minimum width=1.5cm, minimum height=0.3cm, inner sep=0.7mm, anchor=west, below=-0.1mm of dogmm1] (dogmm0) {\bfseries "bud"};
    
    \node[draw, text width=2.1cm, minimum height=0.3cm, inner sep=0.7mm, anchor=west, right=1.5cm of hskymm1] (hskyvt3) {\lstinline[basicstyle=\sffamily\bfseries\small]{Husky::~Husky()}};
    \node[draw, text width=2.1cm, minimum height=0.3cm, inner sep=0.7mm, anchor=west, below=-0.1mm of hskyvt3] (hskyvt2) {\lstinline[basicstyle=\sffamily\bfseries\small]{Husky::sit()}};
    \node[draw, text width=2.1cm, minimum height=0.3cm, inner sep=0.7mm, anchor=west, below=-0.1mm of hskyvt2] (hskyvt1) {\lstinline[basicstyle=\sffamily\bfseries\small]{Husky::bark()}};
    \node[draw, text width=2.1cm, minimum height=0.3cm, inner sep=0.7mm, anchor=west, below=-0.1mm of hskyvt1] (hskyvt0) {\lstinline[basicstyle=\sffamily\bfseries\small]{Dog::getName()}};
    \node[rectangle, text width=2.1cm, minimum height=0.3cm, anchor=west, below=-0.5mm of hskyvt0, xshift=-0.5mm] {vtable Husky};
    
    \node[draw, text width=2.1cm, minimum height=0.3cm, inner sep=0.7mm, anchor=west, below=0.3cm of hskyvt0] (dogvt3) {\lstinline[basicstyle=\sffamily\bfseries\small]{Dog::~Dog()}};
    \node[draw, text width=2.1cm, minimum height=0.3cm, inner sep=0.7mm, anchor=west, below=-0.1mm of dogvt3] (dogvt2) {\lstinline[basicstyle=\sffamily\bfseries\small]{Dog::sit()}};
    \node[draw, text width=2.1cm, minimum height=0.3cm, inner sep=0.7mm, anchor=west, below=-0.1mm of dogvt2] (dogvt1) {\lstinline[basicstyle=\sffamily\bfseries\small]{Dog::bark()}};
    \node[draw, text width=2.1cm, minimum height=0.3cm, inner sep=0.7mm, anchor=west, below=-0.1mm of dogvt1] (dogvt0) {\lstinline[basicstyle=\sffamily\bfseries\small]{Dog::getName()}};
    \node[rectangle, text width=2.1cm, minimum height=0.3cm, anchor=west, below=-0.5mm of dogvt0, xshift=-0.5mm] {vtable Dog};

    \draw[-{Triangle[width=1.5mm, length=1.5mm]}] (petvt0.center) -- (dogmm0.south west);
    \draw[-{Triangle[width=1.5mm, length=1.5mm]}] (petvt1.center) -- (hskymm0.south west);
    \draw[-{Triangle[width=1.5mm, length=1.5mm]}] (dogmm1.center) -- (dogvt0.south west);
    \draw[-{Triangle[width=1.5mm, length=1.5mm]}] (hskymm1.center) -- (hskyvt0.south west);
    
    \node[circle, fill=black, minimum size=1.25mm, inner sep=0mm, outer sep=0mm] at (petvt0.center) {};
    \node[circle, fill=black, minimum size=1.25mm, inner sep=0mm, outer sep=0mm] at (petvt1.center) {};
    \node[circle, fill=black, minimum size=1.25mm, inner sep=0mm, outer sep=0mm] at (dogmm1.center) {};
    \node[circle, fill=black, minimum size=1.25mm, inner sep=0mm, outer sep=0mm] at (hskymm1.center) {};

\end{tikzpicture}
% \columnbreak
    \end{center}

\subsubsection{Redefinition of methods}
    Inherited methods can be redefined in derived classes. If said methods are declared as \textcolor{cactgreen}{\textbf{non-virtual}}, \textcolor{cactgreen}{\textbf{static binding}} is applied and it is determined at \textcolor{cactgreen}{\textbf{compile time}}, whether the base class' or the derived class' method is called. (depending on pointer or reference type)\newline
    If the method is declared as \textcolor{purple}{\textit{\textbf{virtual}}}, \textcolor{purple}{\textit{\textbf{dynamic binding}}} is applied at \textcolor{purple}{\textit{\textbf{runtime}}} and the derived class' method is called, even if the pointer or reference type is of the base class.

    Example:
    \vspace{-2mm}
    \lstinputlisting{snippets/polymorph.h}\vspace{-2mm}
    Classes that are declared as \textit{pure virtual} have no methods defined (method = 0) and cannot be instantiated. They can only be used as base classes for other classes. This is called an interface.

\subsection{Operator Overloading}
    Following operators can be overloaded:
    \vspace{-2mm}
    \lstinputlisting[numbers=none, backgroundcolor=\color{white}]{snippets/overloadable.h}\vspace{-2mm}
    Example:
    \vspace{-2mm}
    \lstinputlisting{snippets/classopov.h}\vspace{-2mm}
