\section{Templates}
\subsection{Function templates}
Function declaration:
\vspace{-2mm}
\lstinputlisting{snippets/funtemplate1.h}\vspace{-2mm}

Function definition:
\vspace{-2mm}
\lstinputlisting{snippets/funtemplate2.h}\vspace{-2mm}
Function templates can be used as \mylstbox{functionName<type1, type2,...>(param1, param2,...);} or \mylstbox{functionName(param1, param2, ...);}. The compiler will deduce the types of the template parameters from the function arguments.

\subsection{Class templates}
Class declaration:
\vspace{-2mm}
\lstinputlisting{snippets/classtemplate1.h}\vspace{-2mm}
Class definition:
\vspace{-2mm}
\lstinputlisting{snippets/classtemplate2.h}\vspace{-2mm}
Class templates can be used as \mylstbox{className<type1, type2,...> objectName;}. The compiler will deduce the types of the template parameters from the object declaration.

\subsection{Rules}
\begin{enumerate}
    \item Templates are \textbf{always} evaluated at \textbf{compile time}.
    \item To instantiate a template, the compiler needs to know following three things:
    \begin{enumerate}
        \item[1.] The template declaration
        \item[2.] The template definition
        \item[3.] Values for the template parameters
    \end{enumerate}
    \item Definition of template functions and methods need to be in the same file as the declaration (.h).
\end{enumerate}